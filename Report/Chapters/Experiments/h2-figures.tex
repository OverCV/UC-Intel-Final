% =============================================================================
% H2: Figuras
% =============================================================================
% INSTRUCCIONES: Coloca tus imágenes en Report/Figures/H2/ con estos nombres:
%   - augmentation_comparison.png  (comparación accuracy/loss con y sin aug)
%   - minority_recall.png          (gráfico de barras recall por clase)
%   - confusion_matrix_aug.png     (opcional: matriz de confusión con aug)

\subsection{Visualización de Resultados}

% Comparación de curvas de entrenamiento
\begin{figure}[htbp]
    \centering
    \includegraphics[width=0.85\textwidth]{Figures/H2/augmentation_comparison.png}
    \caption{Comparación de curvas de entrenamiento con y sin data augmentation. Se observa [TODO: describir diferencias en convergencia y overfitting].}
    \label{fig:h2-augmentation-comparison}
\end{figure}

% Gráfico de recall por clase minoritaria
\begin{figure}[htbp]
    \centering
    \includegraphics[width=0.8\textwidth]{Figures/H2/minority_recall.png}
    \caption{Comparación de recall por clase minoritaria. Las barras azules representan el modelo sin augmentation; las verdes, con augmentation moderada.}
    \label{fig:h2-minority-recall}
\end{figure}

% Matriz de confusión (opcional)
% \begin{figure}[htbp]
%     \centering
%     \includegraphics[width=0.85\textwidth]{Figures/H2/confusion_matrix_aug.png}
%     \caption{Matriz de confusión del modelo entrenado con data augmentation.}
%     \label{fig:h2-confusion-matrix}
% \end{figure}
