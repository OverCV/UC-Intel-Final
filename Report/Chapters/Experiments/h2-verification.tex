% =============================================================================
% H2: Verificación de Hipótesis
% =============================================================================

\subsection{Verificación de Hipótesis H2}

\begin{tcolorbox}[colback=red!5!white, colframe=red!75!black, title=Verificación H2]
\textbf{Hipótesis:} Data augmentation mejorará el recall de clases minoritarias en $\geq$15 puntos porcentuales sin degradar el accuracy global en más de 2\%.

\textbf{Resultados obtenidos:}
\begin{itemize}
    \item Recall promedio de 17 clases minoritarias: 99.4\% $\rightarrow$ 97.1\% ($-$2.3 pp)
    \item Accuracy global: 99.3\% $\rightarrow$ 98.8\% ($-$0.5 pp)
    \item F1-macro: 98.7\% $\rightarrow$ 97.5\% ($-$1.2 pp)
\end{itemize}

\textbf{Conclusión:} \textcolor{red!60!black}{\textbf{HIPÓTESIS RECHAZADA}}

Data augmentation no mejoró el recall de clases minoritarias; por el contrario, lo redujo en $-$2.3 pp. El resultado está muy lejos del umbral esperado de +15 pp. Aunque el impacto en accuracy global ($-$0.5 pp) cumple el criterio de no degradar más de 2\%, la premisa fundamental de mejora en clases minoritarias no se cumplió.
\end{tcolorbox}

\subsubsection{Implicaciones Prácticas}

\begin{itemize}
    \item \textbf{Data augmentation no es universalmente beneficioso:} Para datasets de malware visualizado donde el modelo base ya alcanza alto rendimiento, las técnicas de augmentation pueden ser contraproducentes.

    \item \textbf{La naturaleza del dominio importa:} Las imágenes de malware tienen características estructurales únicas que difieren de imágenes naturales. Las transformaciones geométricas diseñadas para fotografías pueden destruir patrones discriminativos en visualizaciones de binarios.

    \item \textbf{Evaluar antes de aplicar:} En dominios especializados, es esencial evaluar empíricamente el impacto de augmentation en lugar de asumirlo como beneficioso por defecto.

    \item \textbf{Considerar augmentation específica de dominio:} Futuras investigaciones podrían explorar técnicas de augmentation diseñadas específicamente para malware, como perturbaciones a nivel de bytes que preserven la funcionalidad maliciosa.
\end{itemize}
