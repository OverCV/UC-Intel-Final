% =============================================================================
% H3: Análisis de Resultados
% =============================================================================

\subsection{Análisis de Resultados}

\subsubsection{Mejora en Rendimiento}

\begin{itemize}
    \item El F1-score macro [aumentó/disminuyó] de TODO\% (CNN-3) a TODO\% (CNN-5), una diferencia de TODO puntos porcentuales.

    \item El accuracy [aumentó/disminuyó] de TODO\% a TODO\% (+TODO pp).

    \item [TODO: Analizar si la mejora justifica la complejidad adicional].
\end{itemize}

\subsubsection{Costo Computacional}

\begin{itemize}
    \item El tiempo de entrenamiento total [aumentó/disminuyó] de TODO min a TODO min (+TODO\%).

    \item El número de parámetros aumentó de TODO M a TODO M (TODO$\times$).

    \item El tiempo por época [aumentó/disminuyó] de TODO s a TODO s.
\end{itemize}

\subsubsection{Análisis del Trade-off}

\begin{itemize}
    \item \textbf{Rendimiento vs Costo:} Por cada punto porcentual de mejora en F1-score, el tiempo de entrenamiento aumentó TODO\%.

    \item \textbf{Rendimientos decrecientes:} [TODO: Analizar si se observa saturación en el rendimiento].

    \item \textbf{Implicaciones:} [TODO: Discutir cuándo vale la pena usar arquitecturas más profundas].
\end{itemize}
