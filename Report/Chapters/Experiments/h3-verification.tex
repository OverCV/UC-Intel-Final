% =============================================================================
% H3: Verificación de Hipótesis
% =============================================================================

\subsection{Verificación de Hipótesis H3}

% TODO: Cambiar colores según resultado (green=confirmada, yellow=parcial, red=rechazada)
\begin{tcolorbox}[colback=gray!5!white, colframe=gray!75!black, title=Verificación H3]
\textbf{Hipótesis:} Incrementar la profundidad de CNN mejorará el rendimiento, pero con rendimientos decrecientes y mayor costo computacional.

\textbf{Resultados obtenidos:}
\begin{itemize}
    \item F1-score: TODO\% (CNN-3) $\rightarrow$ TODO\% (CNN-5) [+TODO pp]
    \item Accuracy: TODO\% (CNN-3) $\rightarrow$ TODO\% (CNN-5)
    \item Tiempo total: TODO min $\rightarrow$ TODO min [+TODO\%]
    \item Parámetros: TODO M $\rightarrow$ TODO M
\end{itemize}

\textbf{Conclusión:} \textcolor{gray}{\textbf{TODO: CONFIRMADA / PARCIAL / RECHAZADA}}

% TODO: Escribir conclusión basada en resultados
[Descripción de si se observaron rendimientos decrecientes y el trade-off costo/beneficio]
\end{tcolorbox}

\subsubsection{Implicaciones Prácticas}

\begin{itemize}
    \item [TODO: Recomendación sobre profundidad óptima para este tipo de dataset]

    \item [TODO: Cuándo vale la pena usar arquitecturas más profundas]

    \item [TODO: Alternativas como conexiones residuales]
\end{itemize}