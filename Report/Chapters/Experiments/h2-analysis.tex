% =============================================================================
% H2: Análisis de Resultados
% =============================================================================

\subsection{Análisis de Resultados}

\subsubsection{Mejora en Clases Minoritarias}

% TODO: Completar con datos reales
\begin{itemize}
    \item El recall promedio de las clases minoritarias incrementó de TODO\% a TODO\% (+TODO puntos porcentuales).

    \item La clase con mayor mejora fue [TODO: nombre de familia] (+TODO pp), que era la más afectada por el desbalance.

    \item Todas las clases minoritarias mostraron mejoras significativas en recall.
\end{itemize}

\subsubsection{Impacto en Rendimiento Global}

\begin{itemize}
    \item El accuracy global [aumentó/disminuyó] de TODO\% a TODO\% (TODO pp).

    \item El F1-score macro [aumentó/disminuyó] de TODO\% a TODO\%.

    \item El trade-off entre rendimiento global y mejora en minoritarias fue [favorable/desfavorable]: [TODO: explicar].
\end{itemize}

\subsubsection{Observaciones sobre el Entrenamiento}

\begin{itemize}
    \item El modelo con augmentation [convergió más lento/más rápido] debido a la mayor variabilidad en los datos.

    \item El gap train-validation [aumentó/disminuyó], indicando [mayor/menor] capacidad de generalización.

    \item [TODO: Agregar observaciones adicionales sobre curvas de entrenamiento].
\end{itemize}
