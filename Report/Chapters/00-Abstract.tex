\thispagestyle{plain}
\chapter*{Resumen}

% TODO: Completar con los resultados específicos del estudio
Este proyecto aborda el problema crítico de la detección y clasificación automatizada de malware mediante técnicas de Deep Learning. Con el aumento exponencial de amenazas cibernéticas, los métodos tradicionales de detección basados en firmas resultan insuficientes. Proponemos un enfoque basado en visión por computador que convierte ejecutables de malware en imágenes y emplea Redes Neuronales Convolucionales (CNN) para clasificarlos en familias.

La metodología utiliza tres datasets públicos reconocidos: \textit{Blended Malware Image Dataset}, \textit{Malevis Dataset}, y \textit{MalImg Dataset}, que contienen muestras diversas de diferentes familias de malware. Se implementaron y compararon diversas arquitecturas CNN, incluyendo modelos preentrenados y arquitecturas personalizadas, para evaluar su capacidad de generalización y precisión en la clasificación.

Los resultados experimentales demuestran que el enfoque basado en imágenes logra una precisión de clasificación superior al XX\% en la identificación de familias de malware, superando métodos baseline tradicionales. El análisis comparativo revela que las características visuales extraídas por las CNN capturan patrones estructurales distintivos de cada familia, permitiendo una detección robusta incluso ante variantes de malware previamente no vistas.

Esta investigación contribuye al campo de la ciberseguridad proporcionando un método escalable y eficiente para la clasificación automatizada de malware, con aplicaciones potenciales en sistemas de detección de intrusiones y análisis forense digital.

\keywordspt{Malware, Deep Learning, Redes Neuronales Convolucionales, Clasificación de Imágenes, Ciberseguridad, Análisis de Malware.}

\MediaOptionLogicBlank

\pdfbookmark[1]{Abstract}{abstract}
\chapter*{Abstract}

% TODO: Complete with specific study results
This project addresses the critical problem of automated malware detection and classification using Deep Learning techniques. With the exponential increase in cyber threats, traditional signature-based detection methods prove insufficient. We propose a computer vision-based approach that converts malware executables into images and employs Convolutional Neural Networks (CNN) to classify them into families.

The methodology uses three recognized public datasets: \textit{Blended Malware Image Dataset}, \textit{Malevis Dataset}, and \textit{MalImg Dataset}, containing diverse samples from different malware families. Various CNN architectures were implemented and compared, including pre-trained models and custom architectures, to evaluate their generalization capability and classification accuracy.

Experimental results demonstrate that the image-based approach achieves classification accuracy exceeding XX\% in identifying malware families, outperforming traditional baseline methods. Comparative analysis reveals that visual features extracted by CNNs capture distinctive structural patterns of each family, enabling robust detection even against previously unseen malware variants.

This research contributes to the cybersecurity field by providing a scalable and efficient method for automated malware classification, with potential applications in intrusion detection systems and digital forensic analysis.

\keywordsen{Malware, Deep Learning, Convolutional Neural Networks, Image Classification, Cybersecurity, Malware Analysis.}

\MediaOptionLogicBlank
