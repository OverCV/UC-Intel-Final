\chapter{Experimentos y Resultados}
\label{chap:experiments}

Este capítulo presenta los resultados experimentales obtenidos al verificar las tres hipótesis planteadas en el Capítulo \ref{chap:introduction}. Cada sección corresponde a un experimento diseñado específicamente para evaluar una hipótesis, incluyendo configuración, resultados y verificación formal.

% =============================================================================
\section{Configuración Experimental General}
\label{sec:exp:setup}

\subsection{Entorno de Ejecución}

Todos los experimentos se ejecutaron bajo las siguientes condiciones:
\begin{itemize}
    \item \textbf{Hardware:} NVIDIA GPU (especificar modelo), XX GB VRAM
    \item \textbf{Framework:} PyTorch 2.x con CUDA 12.x
    \item \textbf{Reproducibilidad:} Semilla aleatoria fija (seed=42) para todas las ejecuciones
    \item \textbf{Early Stopping:} Paciencia de 10 épocas sobre validation loss
\end{itemize}

\subsection{Dataset: MalImg}

El dataset MalImg contiene imágenes en escala de grises derivadas de ejecutables de malware de Windows. La Tabla \ref{tab:malimg-distribution} muestra la distribución de muestras por familia.

\begin{table}[h]
\centering
\caption{Distribución de familias en el dataset MalImg}
\label{tab:malimg-distribution}
\begin{tabular}{lcc}
\hline
\textbf{Familia} & \textbf{Muestras} & \textbf{Proporción (\%)} \\
\hline
Alueron.gen!J       & 198   & 2.1  \\
C2LOP.P             & 146   & 1.6  \\
C2LOP.gen!g         & 200   & 2.2  \\
Dontovo.A           & 162   & 1.8  \\
Fakerean            & 381   & 4.1  \\
Instantaccess       & 431   & 4.7  \\
Lolyda.AA 1         & 213   & 2.3  \\
Lolyda.AA 2         & 184   & 2.0  \\
Lolyda.AA 3         & 123   & 1.3  \\
Lolyda.AT           & 159   & 1.7  \\
Malex.gen!J         & 136   & 1.5  \\
Obfuscator.AD       & 142   & 1.5  \\
... (continúa)      & ...   & ...  \\
\hline
\textbf{Total}      & \textbf{9,339} & \textbf{100.0} \\
\hline
\end{tabular}
\end{table}

\textbf{Observación sobre desbalance:} Se identifican 5 familias con menos de 100 muestras (clases minoritarias), lo cual motiva la evaluación de data augmentation en H2.

\subsection{Partición de Datos}

\begin{itemize}
    \item \textbf{Entrenamiento:} 70\% (6,537 muestras)
    \item \textbf{Validación:} 15\% (1,401 muestras)
    \item \textbf{Prueba:} 15\% (1,401 muestras)
    \item \textbf{Estratificación:} Sí, manteniendo proporciones por clase
\end{itemize}

% =============================================================================
\section{Experimento 1: Comparación de Arquitecturas (H1)}
\label{sec:exp:h1}

\subsection{Objetivo}

Verificar la hipótesis H1: \textit{``ResNet50 pre-entrenado en ImageNet con fine-tuning superará tanto a una CNN custom como a un Vision Transformer en accuracy y F1-score macro.''}

\subsection{Configuración de Modelos}

Se evaluaron cuatro arquitecturas: un baseline convencional, una CNN profunda de 5 bloques, ResNet50 con fine-tuning, y Vision Transformer.

\subsubsection{Conventional CNN (Baseline)}
\begin{itemize}
    \item \textbf{Arquitectura:} 2 bloques convolucionales (configuración estándar del curso)
    \item \textbf{Estructura:} Conv2D + MaxPool + Conv2D + MaxPool + Flatten + Dense
    \item \textbf{Propósito:} Establecer línea base de comparación
\end{itemize}

\subsubsection{VGG-Mini-H1 (5 bloques)}
\begin{itemize}
    \item \textbf{Arquitectura:} 5 bloques convolucionales (32$\rightarrow$64$\rightarrow$128$\rightarrow$256$\rightarrow$512 filtros)
    \item \textbf{Cada bloque:} Conv2D + BatchNorm + ReLU + MaxPool(2$\times$2)
    \item \textbf{Clasificador:} GlobalAvgPool + Dropout(0.5) + Dense(256) + Dropout(0.3) + Output
\end{itemize}

\subsubsection{ResNet50 (Fine-tuning)}
\begin{itemize}
    \item \textbf{Backbone:} ResNet50 pre-entrenado en ImageNet
    \item \textbf{Estrategia:} Partial fine-tuning (20-30 últimas capas descongeladas)
    \item \textbf{Clasificador:} Dropout(0.5) + Dense(512) + Output
\end{itemize}

\subsubsection{Vision Transformer (ViT-Small)}
\begin{itemize}
    \item \textbf{Patch size:} 16$\times$16
    \item \textbf{Embedding dimension:} 384
    \item \textbf{Depth:} 12 transformer blocks
    \item \textbf{Attention heads:} 6
    \item \textbf{MLP ratio:} 4.0
    \item \textbf{Dropout:} 0.1
\end{itemize}

% --- Archivos modulares ---
% =============================================================================
% H1: Tables - Comparación de Arquitecturas
% =============================================================================

% Tabla principal de resultados
\begin{table}[h]
\centering
\caption{Comparación de rendimiento entre arquitecturas (Experimento H1)}
\label{tab:h1-results}
\begin{tabular}{lccccc}
\hline
\textbf{Modelo} & \textbf{Test Acc} & \textbf{Macro F1} & \textbf{Best Epoch} & \textbf{Tiempo} & \textbf{Esperado} \\
\hline
Conventional CNN (Baseline) & 72.39\%  & 74.01\%  & 9   & 39 min   & --       \\
VGG-Mini-H1 (5 bloques)     & 61.30\%  & N/A      & 10  & 397 min  & $\geq$93\% \\
ViT-Small                   & 74.92\%  & 76.48\%  & 10  & 76 min   & $\geq$91\% \\
\textbf{ResNet50 (Fine-tuned)} & \textbf{96.30\%} & \textbf{95.35\%} & \textbf{6} & \textbf{57 min} & $\geq$96\% \\
\hline
\end{tabular}
\end{table}

% Tabla de hiperparámetros
\begin{table}[h]
\centering
\caption{Hiperparámetros de entrenamiento para Experimento H1}
\label{tab:h1-hyperparams}
\begin{tabular}{lccc}
\hline
\textbf{Parámetro} & \textbf{CNN/Baseline} & \textbf{ResNet50} & \textbf{ViT-Small} \\
\hline
Optimizador        & Adam      & Adam          & AdamW      \\
Learning rate      & 0.001     & 0.0001        & 0.0001     \\
LR Scheduler       & ReduceLROnPlateau & Cosine Annealing & Cosine Annealing \\
Batch size         & 32        & 32            & 32         \\
Épocas máximas     & 100       & 100           & 100        \\
Early stopping     & 15 épocas & 15 épocas     & 20 épocas  \\
Weight Decay       & 0.0001    & 0.0001        & 0.01       \\
\hline
\end{tabular}
\end{table}

% Tabla de métricas detalladas ResNet50
\begin{table}[h]
\centering
\caption{Métricas detalladas de ResNet50 Fine-tuned (mejor modelo)}
\label{tab:h1-resnet-metrics}
\begin{tabular}{lcc}
\hline
\textbf{Métrica} & \textbf{Train} & \textbf{Validation} \\
\hline
Loss (final)       & 0.0378  & 0.1972  \\
Accuracy           & 98.91\% & 97.85\% \\
Precision (macro)  & --      & 95.23\% \\
Recall (macro)     & --      & 95.55\% \\
F1-Score (macro)   & 98.93\% & 95.35\% \\
\hline
\end{tabular}
\end{table}


% =============================================================================
% H1: Figuras
% =============================================================================
% INSTRUCCIONES: Coloca tus imágenes en Report/Figures/H1/ con estos nombres:
%   - resnet50_loss.png
%   - resnet50_accuracy.png
%   - resnet50_f1.png
%   - resnet50_confusion_matrix.png
%   - baseline_curves.png (opcional)

\subsection{Curvas de Entrenamiento}

\begin{figure}[htbp]
    \centering
    \begin{subfigure}[b]{0.48\textwidth}
        \centering
        \includegraphics[width=\textwidth]{Figures/H1/resnet50_loss.png}
        \caption{Loss: Train vs Validation}
        \label{fig:h1-resnet-loss}
    \end{subfigure}
    \hfill
    \begin{subfigure}[b]{0.48\textwidth}
        \centering
        \includegraphics[width=\textwidth]{Figures/H1/resnet50_accuracy.png}
        \caption{Accuracy: Train vs Validation}
        \label{fig:h1-resnet-acc}
    \end{subfigure}
    \caption{Curvas de entrenamiento de ResNet50 Fine-tuned. Se observa convergencia rápida (mejor época: 6) y gap train-val controlado ($\sim$1\%), indicando buena generalización.}
    \label{fig:h1-resnet-curves}
\end{figure}

\begin{figure}[htbp]
    \centering
    \includegraphics[width=0.7\textwidth]{Figures/H1/resnet50_f1.png}
    \caption{Evolución de Precision, Recall y F1-Score durante el entrenamiento de ResNet50. Las tres métricas convergen a valores superiores al 95\%.}
    \label{fig:h1-resnet-prf1}
\end{figure}

% Matriz de confusión
\begin{figure}[htbp]
    \centering
    \includegraphics[width=0.85\textwidth]{Figures/H1/resnet50_confusion_matrix.png}
    \caption{Matriz de confusión de ResNet50 sobre el conjunto de prueba. Los valores en la diagonal representan clasificaciones correctas por familia de malware.}
    \label{fig:h1-confusion-matrix}
\end{figure}

% Comparación visual (opcional)
% \begin{figure}[htbp]
%     \centering
%     \includegraphics[width=0.8\textwidth]{Figures/H1/baseline_curves.png}
%     \caption{Curvas de entrenamiento del modelo baseline (Conventional CNN) para comparación.}
%     \label{fig:h1-baseline-curves}
% \end{figure}


% =============================================================================
% H1: Análisis de Resultados
% =============================================================================

\subsection{Análisis de Resultados}

\subsubsection{ResNet50: Superioridad del Transfer Learning}

ResNet50 con fine-tuning alcanzó el mejor rendimiento con 96.30\% de accuracy y 95.35\% de F1-macro, confirmando la efectividad del transfer learning para datasets de tamaño moderado. Observaciones clave:

\begin{itemize}
    \item \textbf{Convergencia rápida:} Alcanzó su mejor rendimiento en solo 6 épocas, significativamente más rápido que las otras arquitecturas.

    \item \textbf{Overfitting controlado:} El gap entre train (98.91\%) y validation (97.85\%) accuracy fue de apenas 1.06\%, indicando buena generalización.

    \item \textbf{Transferencia efectiva:} Las características de bajo nivel aprendidas en ImageNet (bordes, texturas, patrones) resultaron transferibles al dominio de imágenes de malware.

    \item \textbf{Eficiencia:} Con 57 minutos de entrenamiento, fue más eficiente que ViT (76 min) y dramáticamente más eficiente que la CNN de 5 bloques (397 min).
\end{itemize}

\subsubsection{ViT-Small: Limitaciones con Datasets Pequeños}

Vision Transformer alcanzó 74.92\% de accuracy, por debajo del umbral esperado de 91\%. Este resultado es consistente con la literatura que indica que los transformers requieren datasets significativamente más grandes:

\begin{itemize}
    \item \textbf{Tamaño insuficiente:} MalImg contiene aproximadamente 9,300 muestras, muy por debajo de los millones típicamente requeridos para entrenar transformers desde cero.

    \item \textbf{Falta de pre-entrenamiento:} A diferencia de ResNet50, ViT-Small fue entrenado desde cero, sin aprovechar conocimiento previo.

    \item \textbf{Complejidad del mecanismo de atención:} Los transformers tienen más parámetros a optimizar, lo que dificulta el aprendizaje con datos limitados.
\end{itemize}

\subsubsection{CNN de 5 Bloques: Hallazgo Inesperado}

El resultado más sorprendente fue el bajo rendimiento de la CNN de 5 bloques (61.30\%), significativamente inferior al baseline convencional (72.39\%):

\begin{itemize}
    \item \textbf{Arquitectura sobredimensionada:} 5 bloques convolucionales con progresión 32$\rightarrow$64$\rightarrow$128$\rightarrow$256$\rightarrow$512 filtros resultó excesiva para el dataset.

    \item \textbf{Dificultad de optimización:} El tiempo de entrenamiento extremo (397 minutos) sugiere problemas de convergencia.

    \item \textbf{Posible vanishing gradients:} La profundidad sin conexiones residuales puede haber dificultado el flujo de gradientes.

    \item \textbf{Lección aprendida:} Más profundidad no garantiza mejor rendimiento; la arquitectura debe ser proporcional al tamaño y complejidad del dataset.
\end{itemize}

\subsubsection{Baseline Convencional como Referencia}

El modelo convencional (JorgeNet) con 72.39\% de accuracy proporciona un punto de referencia importante:

\begin{itemize}
    \item Demuestra que arquitecturas simples y bien diseñadas pueden ser competitivas.
    \item Superó a la CNN de 5 bloques, evidenciando que la simplicidad puede ser ventajosa.
    \item El gap de 24 puntos porcentuales con ResNet50 cuantifica el valor del transfer learning.
\end{itemize}


% =============================================================================
% H1: Verificación de Hipótesis
% =============================================================================

\subsection{Verificación de Hipótesis H1}

\begin{tcolorbox}[colback=green!5!white, colframe=green!75!black, title=Verificación H1]
\textbf{Hipótesis:} ResNet50 pre-entrenado superará a CNN custom y ViT-Small en accuracy y F1-score macro.

\textbf{Resultados obtenidos:}
\begin{itemize}
    \item ResNet50 Fine-tuned: 96.30\% accuracy, 95.35\% F1-macro
    \item ViT-Small: 74.92\% accuracy, 76.48\% F1-macro
    \item CNN custom (5 bloques): 61.30\% accuracy
    \item Baseline convencional: 72.39\% accuracy, 74.01\% F1-macro
\end{itemize}

\textbf{Conclusión:} \textcolor{green!60!black}{\textbf{HIPÓTESIS CONFIRMADA}}

ResNet50 superó significativamente a todas las arquitecturas evaluadas, demostrando la efectividad del transfer learning para clasificación de malware visual con datasets de tamaño moderado.
\end{tcolorbox}

\subsubsection{Análisis de Resultados por Arquitectura}

\begin{enumerate}
    \item \textbf{ResNet50 (96.30\%):} El transfer learning desde ImageNet demostró ser altamente efectivo. Las características de bajo nivel (bordes, texturas) aprendidas en imágenes naturales son transferibles al dominio de malware visual.

    \item \textbf{ViT-Small (74.92\%):} El rendimiento limitado es consistente con la literatura: los transformers requieren datasets significativamente más grandes. MalImg ($\sim$9,300 muestras) es insuficiente para aprender representaciones efectivas desde cero.

    \item \textbf{CNN de 5 bloques (61.30\%):} Resultado inesperado. La arquitectura profunda sin conexiones residuales presentó dificultades de optimización, resultando en peor rendimiento que el baseline simple (72.39\%).
\end{enumerate}

\subsubsection{Implicaciones Prácticas}

\begin{itemize}
    \item Para clasificación de malware visual con datasets limitados, \textbf{transfer learning con ResNet50 es la opción recomendada}.

    \item Arquitecturas muy profundas sin skip connections deben evitarse en favor de modelos pre-entrenados.

    \item Vision Transformers requieren datasets significativamente más grandes o pre-entrenamiento específico del dominio.
\end{itemize}


% =============================================================================
\section{Experimento 2: Impacto de Data Augmentation (H2)}
\label{sec:exp:h2}

\subsection{Objetivo}

Verificar la hipótesis H2: \textit{``Data augmentation moderada mejorará significativamente el recall de las familias de malware minoritarias, sin degradar sustancialmente el accuracy global del modelo.''}

\subsection{Configuración}

Se utilizó el mejor modelo de H1 como arquitectura base, entrenando dos versiones: una sin augmentation y otra con augmentation moderada.

% --- Archivos modulares ---
% =============================================================================
% H2: Tables - Impacto de Data Augmentation
% =============================================================================

% Tabla principal de resultados
\begin{table}[h]
\centering
\caption{Impacto de data augmentation en métricas globales}
\label{tab:h2-results}
\begin{tabular}{lcc}
\hline
\textbf{Métrica} & \textbf{Sin Augmentation} & \textbf{Con Augmentation} \\
\hline
Accuracy global              & 99.3\%  & 98.8\%  \\
F1-macro global              & 98.7\%  & 97.5\%  \\
Recall macro                 & 99.5\%  & 97.9\%  \\
Precision macro              & 98.2\%  & 97.2\%  \\
Épocas entrenadas            & 18      & 24      \\
Tiempo de entrenamiento      & $\sim$28 min & $\sim$39 min \\
\hline
\end{tabular}
\end{table}

% Tabla de recall por clase minoritaria (17 clases con ≤30 muestras)
\begin{table}[h]
\centering
\caption{Recall por clase minoritaria antes y después de augmentation (17 clases con $\leq$30 muestras)}
\label{tab:h2-per-class}
\begin{tabular}{lccc}
\hline
\textbf{Familia (Support)} & \textbf{Sin Aug} & \textbf{Con Aug} & \textbf{$\Delta$} \\
\hline
Obfuscator.AD (30)      & 100.0\%  & 100.0\%  & 0 pp \\
Swizzor.gen!E (30)      & 100.0\%  & 90.0\%   & $-$10.0 pp \\
Alueron.gen!J (29)      & 100.0\%  & 100.0\%  & 0 pp \\
C2LOP.gen!g (28)        & 100.0\%  & 100.0\%  & 0 pp \\
Rbot!gen (26)           & 100.0\%  & 92.3\%   & $-$7.7 pp \\
Dontovo.A (26)          & 100.0\%  & 100.0\%  & 0 pp \\
Autorun.K (21)          & 100.0\%  & 100.0\%  & 0 pp \\
Malex.gen!J (20)        & 95.0\%   & 95.0\%   & 0 pp \\
Lolyda.AA2 (20)         & 100.0\%  & 100.0\%  & 0 pp \\
Lolyda.AA3 (19)         & 94.7\%   & 94.7\%   & 0 pp \\
Lolyda.AT (19)          & 100.0\%  & 100.0\%  & 0 pp \\
C2LOP.P (17)            & 100.0\%  & 94.1\%   & $-$5.9 pp \\
Skintrim.N (17)         & 100.0\%  & 100.0\%  & 0 pp \\
Adialer.C (16)          & 100.0\%  & 100.0\%  & 0 pp \\
Swizzor.gen!I (13)      & 100.0\%  & 84.6\%   & $-$15.4 pp \\
Agent.FYI (12)          & 100.0\%  & 100.0\%  & 0 pp \\
Wintrim.BX (10)         & 100.0\%  & 100.0\%  & 0 pp \\
\hline
\textbf{Promedio (17 clases)} & \textbf{99.4\%} & \textbf{97.1\%} & \textbf{$-$2.3 pp} \\
\hline
\end{tabular}
\end{table}

% Tabla de configuración de augmentation
\begin{table}[h]
\centering
\caption{Configuración de data augmentation aplicada}
\label{tab:h2-augmentation-config}
\begin{tabular}{lc}
\hline
\textbf{Transformación} & \textbf{Parámetros} \\
\hline
Rotaciones ortogonales & 90°, 180°, 270° (selección aleatoria) \\
Flip horizontal        & probabilidad 0.5 \\
Flip vertical          & probabilidad 0.5 \\
Variación de brillo    & $\pm$20\% \\
Variación de contraste & $\pm$20\% \\
\hline
\end{tabular}
\end{table}


% =============================================================================
% H2: Figuras
% =============================================================================
% INSTRUCCIONES: Coloca tus imágenes en Report/Figures/H2/ con estos nombres:
%   - augmentation_comparison.png  (comparación accuracy/loss con y sin aug)
%   - minority_recall.png          (gráfico de barras recall por clase)
%   - confusion_matrix_aug.png     (opcional: matriz de confusión con aug)

\subsection{Visualización de Resultados}

% Comparación de curvas de entrenamiento
\begin{figure}[htbp]
    \centering
    \includegraphics[width=0.85\textwidth]{Figures/H2/augmentation_comparison.png}
    \caption{Comparación de curvas de entrenamiento con y sin data augmentation. Se observa [TODO: describir diferencias en convergencia y overfitting].}
    \label{fig:h2-augmentation-comparison}
\end{figure}

% Gráfico de recall por clase minoritaria
\begin{figure}[htbp]
    \centering
    \includegraphics[width=0.8\textwidth]{Figures/H2/minority_recall.png}
    \caption{Comparación de recall por clase minoritaria. Las barras azules representan el modelo sin augmentation; las verdes, con augmentation moderada.}
    \label{fig:h2-minority-recall}
\end{figure}

% Matriz de confusión (opcional)
% \begin{figure}[htbp]
%     \centering
%     \includegraphics[width=0.85\textwidth]{Figures/H2/confusion_matrix_aug.png}
%     \caption{Matriz de confusión del modelo entrenado con data augmentation.}
%     \label{fig:h2-confusion-matrix}
% \end{figure}


% =============================================================================
% H2: Análisis de Resultados
% =============================================================================

\subsection{Análisis de Resultados}

\subsubsection{Efecto en Clases Minoritarias}

Contrariamente a lo esperado, data augmentation no mejoró el recall de las clases minoritarias, sino que lo deterioró:

\begin{itemize}
    \item El recall promedio de las 17 clases minoritarias ($\leq$30 muestras) \textbf{disminuyó} de 99.4\% a 97.1\% ($-$2.3 puntos porcentuales).

    \item De las 17 clases minoritarias, 4 experimentaron degradación: \textbf{Swizzor.gen!I} ($-$15.4 pp), \textbf{Swizzor.gen!E} ($-$10.0 pp), \textbf{Rbot!gen} ($-$7.7 pp), y \textbf{C2LOP.P} ($-$5.9 pp).

    \item Las restantes 13 clases minoritarias mantuvieron su rendimiento sin cambios significativos.
\end{itemize}

\subsubsection{Impacto en Rendimiento Global}

\begin{itemize}
    \item El accuracy global \textbf{disminuyó} de 99.3\% a 98.8\% ($-$0.5 pp).

    \item El F1-score macro \textbf{disminuyó} de 98.7\% a 97.5\% ($-$1.2 pp).

    \item El recall macro \textbf{disminuyó} de 99.5\% a 97.9\% ($-$1.6 pp).

    \item Todas las métricas mostraron degradación con augmentation, indicando que las transformaciones aplicadas introdujeron ruido perjudicial.
\end{itemize}

\subsubsection{Análisis de Causas}

El resultado negativo de data augmentation puede explicarse por varios factores:

\begin{enumerate}
    \item \textbf{Rendimiento base excepcionalmente alto:} El modelo sin augmentation ya alcanzaba 99.4\% de recall promedio en clases minoritarias, dejando poco margen de mejora.

    \item \textbf{Naturaleza estructural de imágenes de malware:} A diferencia de imágenes naturales, las visualizaciones de malware representan bytes de ejecutables donde cada píxel tiene significado semántico. Las transformaciones geométricas pueden alterar patrones discriminativos específicos de cada familia.

    \item \textbf{Sensibilidad variable entre familias:} Las familias Swizzor (gen!I y gen!E), Rbot!gen, y C2LOP.P mostraron mayor sensibilidad a las transformaciones, posiblemente porque sus patrones visuales distintivos se ven más afectados por rotaciones y volteos.

    \item \textbf{Overfitting a variantes artificiales:} El modelo con augmentation entrenó más épocas (24 vs 18) pero con datos transformados que podrían no representar variantes reales de malware.
\end{enumerate}

\subsubsection{Observaciones sobre el Entrenamiento}

\begin{itemize}
    \item El modelo con augmentation requirió más épocas para converger (24 vs 18), indicando mayor dificultad de optimización.

    \item A pesar del entrenamiento más prolongado, el modelo con augmentation no logró igualar el rendimiento del modelo base.

    \item El gap entre train loss y val loss fue similar en ambos casos, sugiriendo que el problema no es overfitting sino degradación de la señal discriminativa.
\end{itemize}


% =============================================================================
% H2: Verificación de Hipótesis
% =============================================================================

\subsection{Verificación de Hipótesis H2}

% TODO: Cambiar colores según resultado (green=confirmada, yellow=parcial, red=rechazada)
\begin{tcolorbox}[colback=gray!5!white, colframe=gray!75!black, title=Verificación H2]
\textbf{Hipótesis:} Data augmentation mejorará significativamente el recall de clases minoritarias sin degradar sustancialmente el accuracy global.

\textbf{Resultados obtenidos:}
\begin{itemize}
    \item Recall promedio minoritarias: TODO\% $\rightarrow$ TODO\% (+TODO pp)
    \item Accuracy global: TODO\% $\rightarrow$ TODO\% (TODO pp)
    \item F1-macro: TODO\% $\rightarrow$ TODO\%
\end{itemize}

\textbf{Conclusión:} \textcolor{gray}{\textbf{TODO: CONFIRMADA / PARCIAL / RECHAZADA}}

% TODO: Escribir conclusión basada en resultados
[Descripción de por qué la hipótesis se confirma o rechaza]
\end{tcolorbox}

\subsubsection{Implicaciones Prácticas}

\begin{itemize}
    \item [TODO: Implicación 1 sobre cuándo usar augmentation]

    \item [TODO: Implicación 2 sobre configuración recomendada]

    \item [TODO: Implicación 3 sobre limitaciones]
\end{itemize}




% =============================================================================
\section{Experimento 3: Efecto de la Profundidad en CNN (H3)}
\label{sec:exp:h3}

\subsection{Objetivo}

Verificar la hipótesis H3: \textit{``Incrementar la profundidad de una CNN custom mejorará el rendimiento del modelo, pero con rendimientos decrecientes y mayor costo computacional.''}

\subsection{Configuración de Arquitecturas}

Se compararon dos arquitecturas CNN con diferente profundidad, manteniendo la misma estructura de bloque (Conv + BN + ReLU + MaxPool) y clasificador.


Para abordar esta hipótesis, se diseñó un enfoque metodológico sistemático que involucra:
\begin{enumerate}
    \item Diseño de dos arquitecturas CNN con profundidad incremental pero manteniendo coherencia estructural
    \item Entrenamiento bajo condiciones controladas (mismo optimizador, épocas, conjunto de datos)
    \item Evaluación exhaustiva mediante métricas múltiples (accuracy, loss, F1, brecha de generalización)
    \item Análisis comparativo de curvas de aprendizaje y estabilidad del entrenamiento
    \item Medición indirecta del costo computacional mediante inferencia estructural
\end{enumerate}

La lógica experimental se basa en el principio de \textit{ceteris paribus} (todo lo demás constante), donde solo la profundidad de la red varía entre condiciones, permitiendo aislar su efecto específico sobre el rendimiento y eficiencia.



\begin{figure}[h]
\centering
\includegraphics[width=0.45\textwidth,height=5cm]{Figures/H3/c1.jpg}
\includegraphics[width=0.45\textwidth,height=5cm]{Figures/H3/c2.jpg}
\caption{Configuración para los diferentes modelos}
\label{fig:curves-exp3}
\end{figure}


\subsection{Configuración Entrenamiento}

\subsubsection{Especificaciones Técnicas Detalladas}

Las arquitecturas comparadas comparten una filosofía de diseño modular basada en bloques convolucionales secuenciales, pero difieren en el número de estos bloques. La Tabla \ref{tab:h3-architectures-detailed} presenta una descomposición técnica exhaustiva.

\begin{table}[h]
\centering
\caption{Análisis técnico detallado de las arquitecturas CNN comparadas}
\label{tab:h3-architectures-detailed}
\begin{tabularx}{\textwidth}{|l|X|X|}
\hline
\textbf{Aspecto} & \textbf{H2\_MOD.A (9 capas)} & \textbf{Arquitectura de 12 Capas} \\
\hline
\textbf{Filosofía de Diseño} & 
Arquitectura moderadamente profunda con 2 bloques convolucionales. Enfoque en extracción jerárquica de características sin complejidad excesiva. &
Extensión profunda con 3 bloques convolucionales para mayor capacidad de representación. 
\\ \hline

\textbf{Flujo de Datos} &
Entrada $\rightarrow$ Bloque1(Conv+Conv+Pool) $\rightarrow$ Bloque2(Conv+Conv+Pool) $\rightarrow$ Regularización(Dropout) $\rightarrow$ Clasificador(Flatten+Dense) $\rightarrow$ Salida &
Entrada $\rightarrow$ Bloque1(Conv+Conv+Pool) $\rightarrow$ Bloque2(Conv+Conv+Pool) $\rightarrow$ Bloque3(Conv+Conv+Pool) $\rightarrow$ Regularización(Dropout) $\rightarrow$ Clasificador(Flatten+Dense) $\rightarrow$ Salida
\\ \hline

\textbf{Capacidad de Representación} & 210,000 parámetros entrenables aprox. & 280,000 parámetros entrenables aprox. (+33\%). 
\\ \hline

\textbf{Regularización} & Dropout único del 25\%. & Dropout del 25\% aplicado al tercer bloque. 
\\ \hline

\textbf{Hiperparámetros Comunes} &
Optimizador Adam (lr=0.001), 10 épocas, Batch Size 32, activación ReLU, función de pérdida Categorical Crossentropy. &
Idéntica configuración.
\\ \hline

\end{tabularx}
\end{table}


\subsubsection{Consideraciones de Implementación}

Ambos modelos fueron implementados en TensorFlow 2.x con inicialización He normal. El preprocesamiento incluyó normalización de píxeles (0-1) sin aumentación para aislar el efecto de la profundidad. Entrenamiento en GPU Nvidia T4 (16GB), verificando uso de memoria para evitar limitaciones.

\subsection{Resultados Cuantitativos y Comparación Exhaustiva}

\subsubsection{Métricas de Rendimiento Principal}

\begin{table}[h]
\centering
\caption{Análisis cuantitativo detallado del rendimiento por arquitectura}
\label{tab:h3-results-detailed}
\resizebox{\textwidth}{!}{%
\begin{tabular}{|p{3.5cm}|c|c|c|p{6cm}|}
\hline
\textbf{Categoría} & \textbf{Métrica} & \textbf{H2\_MOD.A (9 capas)} & \textbf{12 Capas} & \textbf{Interpretación} \\ \hline

\multirow{2}{*}{Precisión} 
& Val Accuracy (\%) & \textbf{85.29} & 83.45 & Disminución con mayor profundidad \\ \cline{2-5}
& Test Accuracy (\%) & N/D & \textbf{85.58} & Generaliza moderadamente bien \\ \hline

\multirow{2}{*}{Pérdida}
& Val Loss & \textbf{0.3677} & 0.4095 & Mayor error residual en modelo profundo \\ \cline{2-5}
& Train Loss & 0.2644 & 0.3061 & Dificultad de optimización \\ \hline

\multirow{3}{*}{Generalización}
& Brecha (pp) & 2.86 & 4.13 & Aumento de sobreajuste \\ \cline{2-5}
& Ratio Train/Val Loss & 1.39 & 1.33 & Menor estabilidad \\ \cline{2-5}
& Consistencia Épocas Finales & Alta & Media & Volatilidad \\ \hline

\multirow{3}{*}{Métricas por Clase}
& Macro F1 (\%) & N/D & 83.54 & Rendimiento no equilibrado \\ \cline{2-5}
& Weighted F1 (\%) & N/D & 85.54 & Dominado por clases mayoritarias \\ \cline{2-5}
& Desviación Std F1 & N/D & $\sim$8.2 & Variabilidad entre clases \\ \hline

\multirow{2}{*}{Eficiencia}
& Duración Entrenamiento & 33m 48s & $\sim$45m & +33\% estimado \\ \cline{2-5}
& Memoria GPU & $\sim$3.2GB & $\sim$4.1GB & +28\% estimado \\ \hline

\end{tabular}
}
\end{table}


\subsubsection{Curvas de Aprendizaje}

\begin{figure}[h]
    \centering
    \begin{subfigure}{0.48\textwidth}
        \centering
        \caption{Curvas de Loss: mayor volatilidad en 12 capas}
    \end{subfigure}
    \hfill
    \begin{subfigure}{0.48\textwidth}
        \centering
        \caption{Curvas de Accuracy: convergencia inferior en modelo profundo}
    \end{subfigure}
    \caption{Comparación visual de dinámicas de aprendizaje}
\end{figure}


\begin{figure}[h]
\centering
\includegraphics[width=0.45\textwidth,height=5cm]{Figures/H3/ima2.jpg}
\includegraphics[width=0.45\textwidth,height=5cm]{Figures/H3/ima2.jpg}
\caption{Comparación de curvaturas de entrenamiento para el Experimento: Accuracy (izquierda) y Loss (derecha).}
\label{fig:curves-exp3}
\end{figure}

\subsubsection{Análisis de la Brecha de Generalización}

La brecha se incrementó 44.4\% (2.86pp a 4.13pp), atribuible a:

\begin{itemize}
    \item Relación parámetros/datos más desfavorable
    \item Desvanecimiento parcial de gradientes sin conexiones residuales
    \item Co-adaptación de características con menor generalización
\end{itemize}

\subsection{Análisis de la Hipótesis por Componente}

\subsubsection{Rendimientos Decrecientes}

\begin{enumerate}
    \item El accuracy disminuyó (85.29\% $\rightarrow$ 83.45\%).
    \item Aumento del 33\% en parámetros no mejoró el rendimiento.
    \item Rentabilidad marginal negativa: -0.0055\% por 1000 parámetros.
\end{enumerate}

\subsubsection{Costo Computacional}

\begin{itemize}
    \item Forward Pass:
    \begin{align*}
        \text{H2\_MOD.A} &: \sim 85 \text{ MFLOPs} \\
        \text{12 Capas} &: \sim 115 \text{ MFLOPs}
    \end{align*}
    \item Memoria:
    \begin{align*}
        45\text{MB} \rightarrow 62\text{MB}
    \end{align*}
    \item Tiempo por época:
    \begin{align*}
        3.38 \text{ min} \rightarrow 4.57 \text{ min}
    \end{align*}
\end{itemize}

\subsection{Verificación Sistemática de H3}

\begin{table}[h]
\centering
\caption{Verificación cuantitativa de componentes de H3}
\label{tab:h3-hypothesis-verification}
\begin{tabular}{|p{6cm}|c|p{7cm}|}
\hline
\textbf{Componente} & \textbf{Verificación} & \textbf{Evidencia} \\ \hline
``Mayor profundidad mejora rendimiento'' & Rechazada & Accuracy menor, pérdida mayor \\
``Rendimientos decrecientes'' & Confirmada & Más recursos no $\rightarrow$ mejor resultado \\
``Mayor costo computacional'' & Confirmada & +35\% FLOPs, +33\% tiempo, +28\% memoria \\
``Mayor capacidad de patrones'' & Parcial & Test Accuracy positivo pero F1 desigual \\
\hline
\end{tabular}
\end{table}





\subsubsection{Conclusión Final}

Este experimento demuestra que incrementar la profundidad sin técnicas complementarias no garantiza mayor rendimiento. La profundidad aporta capacidad, pero también costos y riesgo de saturación temprana. La efectividad se encuentra en un balance entre profundidad, generalización y eficiencia computacional.



